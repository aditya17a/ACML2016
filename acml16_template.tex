%\documentclass[wcp,gray]{jmlr} % test grayscale version
\documentclass[wcp]{jmlr}

% The following packages will be automatically loaded:
% amsmath, amssymb, natbib, graphicx, url, algorithm2e

%\usepackage{rotating}% for sideways figures and tables
%\usepackage{longtable}% for long tables
\usepackage{hyperref}
\usepackage[section]{placeins}
%\usepackage{url}
\usepackage{physics}
%\usepackage{amsmath}
\usepackage{graphicx,import}
%\usepackage[norelsize]{algorithm2e}
% The booktabs package is used by this sample document
% (it provides \toprule, \midrule and \bottomrule).
% Remove the next line if you don't require it.
%\usepackage{booktabs}
% The siunitx package is used by this sample document
% to align numbers in a column by their decimal point.
% Remove the next line if you don't require it.
%\usepackage[load-configurations=version-1]{siunitx} % newer version
%\usepackage{siunitx}
%\usepackage{natbib}

% The following command is just for this sample document:
%\newcommand{\cs}[1]{\texttt{\char`\\#1}}

\jmlrvolume{60}
\jmlryear{2016}
\jmlrworkshop{ACML 2016}

\title[The Incredible Shrinking Neural Network]{The Incredible Shrinking Neural Network: Pruning to Operate in Constrained Memory Environments}


 % Authors with different addresses:
%  \author{\Name{Nikolas Wolfe} \Email{nwolfe@cs.cmu.edu}\\
%  \Name{Aditya Sharma} \Email{adityasharma@cmu.edu}\\
%  \Name{Bhiksha Raj} \Email{bhiksha@cs.cmu.edu}\\
%  \addr Carnegie Mellon University
% }

%\editors{List of editors' names}

\newcommand{\fix}{\marginpar{FIX}}
\newcommand{\new}{\marginpar{NEW}}
\newcommand{\Out}[2]{o_{#1}^{(#2)}}
\newcommand{\Target}[1]{t_{#1}}
\newcommand{\Input}[2]{x_{#1}^{(#2)}}
\newcommand{\Weight}[3]{w_{#1#2}^{(#3)}}
\newcommand{\Con}[3]{c_{#1#2}^{(#3)}}
\newcommand{\Etotal}{E_{\mathrm{total}}}

\setlength{\parindent}{0pt}

\begin{document}

\maketitle

\input{abstract.tex}
\begin{keywords}
pruning, neural network compression, pruning neurons, learning representation
\end{keywords}

\input{introduction.tex}
\input{literature_review.tex}
\input{methodology.tex}
%\input{experiments.tex}
\input{results.tex}
%\input{discussion.tex}
\input{conclusions.tex}
%\input{acknowledgments.tex}



%\acks{We would like to thank Lukas Drude for the assistance with calculations on collecting quadratic error gradients through the 2nd order version of back-propagation. We would also like to thank Rita Singh, Abelino Jiminez, Thomas Schaaf and Don Wolfe for helpful discussions.}

%\bibliographystyle{plain}
\bibliography{acml16}

%\appendix

%\section{First Appendix}\label{apd:first}

%\input{first_second_derivatives.tex}

%\input{first_second_derivatives.tex}
%\input{diagram.tex}

%\section{Second Appendix}\label{apd:second}

%This is the second appendix.


\end{document}
